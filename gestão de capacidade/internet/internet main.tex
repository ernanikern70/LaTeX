% PREÂMBULO

% ---
% Opções do documento
% ---
\documentclass[a4paper,10pt]{article}

% ---
% Pacotes 
% ---
\usepackage[utf8]{inputenc} % pacote para acentuação
\usepackage[T1]{fontenc}
\usepackage[brazil]{babel} % coloca os nomes em pt-br
\usepackage{indentfirst} % aplica indentação
\usepackage{setspace} % altera espaçamento entre linhas
%\usepackage[a4paper, margin=2cm]{geometry} % ajusta a margem geral
\usepackage[a4paper, left=2cm, right=1.5cm, top=3cm, bottom=2.5cm]{geometry} % adiciona o pacote 'geometry' e configura margens
\usepackage{xcolor} % altera as cores do documento
\usepackage{graphicx} % permite adicionar figuras, sem forçar posição
\usepackage{float} % insere figuras forçando o posicionamento
\usepackage{tabularx} % ajusta margens
\usepackage{amsmath} % modo matemático
\usepackage{draftwatermark} % habilita marca dágua
\usepackage{transparent} % transparência
\usepackage{enumitem} % disponibiliza opções para listas
\usepackage{tocloft}
\usepackage{titling} % controla espaçamentos do /maketitle
\usepackage[colorlinks=true,linkcolor=blue,urlcolor=blue,citecolor=red]{hyperref} % ver comentários abaixo
%\usepackage[hidelinks]{hyperref} % links http - MANTER ABAIXO DOS OUTROS 'usepackage', pois pode alterar alguns comandos. - HIDELINKS evita que os títulos do sumário e notas de rodapé recebam um círculo colorido
%\usepackage{colortbl}
%\usepackage{booktabs}
%\usepackage{multirow} % mescla linhas de tabelas
%\usepackage{multicol} % usar texto em colunas
%\usepackage{csquotes} % coloca aspas para citações diretas

\setlength{\parindent}{1,5cm} % ajusta a indentação do parágrafo
\setlength{\parskip}{0.1cm} % ajusta a separação entre parágrafos
\setlength{\columnsep}{5mm} % separação entre colunas
\setlist{itemsep=-5pt, topsep=0pt} % define espaçamento de listas globalmente

% deixa todo o documento em sans serif
\renewcommand{\familydefault}{\sfdefault}

% configura espaçamentos do \maketitle
\pretitle{\begin{center}\LARGE\bfseries}
\posttitle{\par\vspace{5cm}\end{center}}
\preauthor{\begin{center}\large}
\postauthor{\par\vspace{3cm}\end{center}}
\predate{\begin{center}\normalsize}
\postdate{\par\end{center}}

% ---
% CORPO DO TEXTO
% ---
\begin{document}
\SetWatermarkText{RASCUNHO} 
\SetWatermarkScale{3}
\sffamily

% ---
% CAPA
% ---
	\title{\textbf{\huge{Gestão de Capacidade - INTERNET}}}
	\author{\textbf{Ernani Kern - DIRT}}

	\maketitle % mostra os parâmetros acima, senão, não aparecem
	\thispagestyle{empty} % oculta a numeração da página
	\newpage
	\setcounter{page}{1} % reseta a contagem de páginas, e inicia aqui com '1'
	\pagenumbering{Roman} % altera para alg romano; o 'R' maiúsculo deixa a numeração em maiúsculas
	\tableofcontents % cria o sumário
	\listoffigures % cria sumário de figuras
	\newpage
	
	\setcounter{page}{1} % reinicia de novo a numeração de páginas
	\pagenumbering{arabic} % numeração de páginas em arábico
	\onehalfspacing % espaçamento 1,5 entre linhas
	
	\section{Canais considerados}

	\begin{itemize}
		\item Ávato AS262907: dois canais de 2Gbps
		\item Adylnet AS28282: dois canais de 2Gbps
	\end{itemize}	

	
	\section{Canais não considerados}
	\begin{itemize}
		\item IX.br de Porto Alegre AS26162: uma fibra óptica apagada, atualmente de 10Gbps. 
	\end{itemize}
	
		Esse canal permite a troca de tráfego com os demais parceiros do IX.br conectados no ponto de troca de tráfego de Porto Alegre. Por ser uma fibra óptica apagada, a velocidade está limitada à capacidade do hardware utilizado na ponta do TRT4 e na ponta do IX.br. 
	
	\section{Tráfego de download no DC1}
	Abaixo temos alguns gráficos dos tráfegos registrados no sistema de monitoração Zabbix (entre setembro de 2024 e maio de 2025 - cada operadora é mostrada duas vezes em cada gráfico devido à mudança dos roteadores Alpha em dezembro/24):
	
	\begin{figure}[H]
		\centering
		\includegraphics[width=1\linewidth]{images/dc1 download.png}
		\caption[DC1 Download]{DC1 Download 08-19 hs}
		\label{fig:dc1d}
	\end{figure}
	
	Acima, a operadora Adylnet teve tráfego máximo de 67,5\% da banda em 22/10/2024, e no restante do período se manteve abaixo de 57\% de uso do link. 
	A operadora Ávato teve um pico de 50\% em 10/12 e no restante do período não ultrapassou 36\% de uso do link.
	
	\section{Tráfego de upload do DC1}
	
	\begin{figure}[H]
		\centering
		\includegraphics[width=1\linewidth]{images/dc1 upload.png}
		\caption[DC1 Upload]{DC1 Upload 08-19hs}
		\label{fig:dc1u}
	\end{figure}
	
	No upload a operadora Adylnet atingiu em 11/11/2024 o pico de 84\% de utilização, e se manteve abaixo de 71\% no restante do período. 
	A Ávato operou abaixo de 50\% de ocupação em todo o período.
	
	\section{Tráfego de download do DC2}
	
	\begin{figure}[H]
		\centering
		\includegraphics[width=1\linewidth]{images/dc2 download.png}
		\caption[DC2 Download]{DC2 Download 08-19 hs}
		\label{fig:dc2d}
	\end{figure}

	Acima, a Adylnet teve apenas um pico de 20\% de ocupação em 04/10/2024, e no restante do período teve pouco uso. 
	A Ávato apresentou picos entre 43\% e 35\% entre 02/10/2024 e 14/12/2024, e no restante do período permaneceu abaixo de 32\% de ocupação do link.
	
	\section{Tráfego de upload do DC2}
	
	\begin{figure}[H]
		\centering
		\includegraphics[width=1\linewidth]{images/dc2 upload.png}
		\caption[DC2 Upload]{DC2 Upload 08-19 hs}
		\label{fig:dc2u}
	\end{figure}
	
	Neste gráfico o maior tráfego na operadora Adylnet foi de 63\% da capacidade em 04/10/2024; nos demais períodos teve outro pico de 50\% em 10/12/2024 e no restante do tempo se manteve abaixo de 31,5\% de tráfego máximo. 
	
	A Ávato teve dois picos de tráfego em 11/09/2024 e 16/04/2025, ambos de 53,5\%, e no restante do tempo se manteve abaixo de 45\% de tráfego máximo. 
	
	\section{Conclusão}
	
	Conforme visualizado nos gráficos, as operadoras de Internet apresentam apenas alguns picos de utilização em datas isoladas, mantendo uma média de tráfego bem abaixo de suas capacidades atuais, o que não indica necessidade de aumento de links. 
	
\end{document}
