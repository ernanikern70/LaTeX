% PREÂMBULO
\documentclass[a4paper,10pt]{article}

\usepackage[utf8]{inputenc} % pacote para acentuação
\usepackage[brazil]{babel} % coloca os nomes em pt-br
\usepackage{indentfirst} % aplica indentação
\usepackage{setspace} % altera espaçamento entre linhas
%\usepackage[a4paper, margin=2cm]{geometry} % ajusta a margem geral
\usepackage[a4paper, left=2cm, right=1.5cm, top=3cm, bottom=2cm]{geometry} % adiciona o pacote 'geometry' e configura margens
\usepackage{xcolor} % altera as cores do documento
\usepackage{graphicx} % permite adicionar figuras, sem priorizar a posição
\usepackage{float} % insere figuras forçando o posicionamento
\usepackage{colortbl}
\usepackage{booktabs}
\usepackage{multirow} % mescla linhas de tabelas
\usepackage{tabularx} % ajusta margens
\usepackage{amsmath} % modo matemático

\setlength{\parindent}{1,5cm} % ajusta a indentação do parágrafo
\setlength{\parskip}{1cm} % ajusta a separação entre parágrafos

\date{27/11/2025}

\renewcommand{\sin}{\mathrm{sen\hspace{0.5mm}}} % traduz a função 'sin' para o português
%\DeclareMathOperator{\sen}{sen} % método melhor que o acima, mas exige que se use '\sen' no texto

% CORPO DO TEXTO
\begin{document}
\sffamily
\title{\textbf{\huge{Capa e Sumário}}}
\author{\textbf{Ernani Kern}}
\date{\textbf{Dezembro de 2025}}
\maketitle % mostra os parâmetros acima, senão, não aparecem
\thispagestyle{empty} % oculta a numeração da página
\newpage

\setcounter{page}{1} % reseta a contagem de páginas, e inicia aqui com '1'
\pagenumbering{Roman} % altera para alg romano; o 'R' maiúsculo deixa a numeração em maiúsculas
\tableofcontents % cria o sumário
\newpage

\listoffigures % cria sumário de figuras
\newpage

\listoftables % cria a lista de tabelas
\newpage

\setcounter{page}{1} % reinicia de novo a numeração de páginas
\pagenumbering{arabic} % numeração de páginas em arábico
    \section{Seção antes da seção 1}
    \onehalfspacing % espaçamento 1,5 entre linhas
No mundo atual, a competitividade nas transações comerciais ressalta a relevância da participação ativa da articulação interinstitucional necessária. Considerando as lições aprendidas, a valorização de fatores subjetivos não pode mais se dissociar de alternativas às soluções ortodoxas. Não obstante, a expansão dos mercados mundiais apresenta tendências no sentido de aprovar a manutenção das condições inegavelmente apropriadas. O empenho em analisar a necessidade de renovação processual legitima a busca por soluções sistêmicas dos níveis de motivação departamental.
\textcolor{red}{Texto em vermelho}
\colorbox{gray}{\textcolor{blue}{Texto Azul e Fundo cinza}}
%\pagecolor{cyan}

\section{Seção 1}
\setstretch{2} % ajusta espaçamento personalizado
É importante questionar o quanto a adoção de políticas descentralizadoras ainda não demonstrou convincentemente que vai participar na mudança das rupturas provocadas pela transformação digital. As experiências acumuladas demonstram que o desenvolvimento contínuo de distintas formas de atuação modifica os parâmetros tradicionais de análise dos conhecimentos estratégicos para atingir a excelência.
% para abrir parágrafo, deixar linha anterior em branco

\begin{onehalfspace} % \begin{spacing}{1,5}
Evidentemente, a complexidade dos estudos efetuados deve passar por modificações independentemente do fluxo de informações. À luz das boas práticas institucionais, a consulta aos diversos militantes é uma das consequências dos instrumentos regulatórios vigentes.
\end{onehalfspace} % \end{spacing}

    \subsection{Subseção x.x}
Vale destacar que o início da atividade geral de formação de atitudes exige a precisão e a definição das novas proposições. Por intermédio de análises qualificadas, a valorização da diversidade de pensamento talvez venha a ressaltar a relatividade das condições estruturais subjacentes.\\ De acordo com especialistas, a crescente influência da mídia faz parte de um processo de gerenciamento das regras de conduta normativas.\\ Ainda assim, existem dúvidas a respeito de como a revolução dos costumes agrega valor ao estabelecimento dos aprendizados oriundos da experiência acumulada.
% o '\\' gera quebra de linha
    \subsubsection*{Subseção x.x.x}
A prática cotidiana prova que o aumento do diálogo entre os diferentes setores produtivos nos obriga à análise do remanejamento dos quadros funcionais. Pensando mais a longo prazo, a hegemonia do ambiente político oferece uma interessante oportunidade para verificação do processo de comunicação como um todo. Percebemos, cada vez mais, que a redefinição dos modelos de governança aprofunda o debate sobre a sustentabilidade dos processos que moldam a realidade institucional.
% o '*' na '\subsubsection' elimina a numeração da seção; se houver nova seção após, a numeração continuará a partir da última numerada.

    \subsubsection{Subseção x.x.x}
Podemos já vislumbrar o modo pelo qual a expansão dos canais de diálogo social auxilia a preparação e a composição das formas de ação. Assim mesmo, a estruturação de redes de cooperação intersetorial aponta para a melhoria da governabilidade em ambientes voláteis. Neste sentido, o julgamento imparcial das eventualidades maximiza as possibilidades por conta da gestão inovadora da qual fazemos parte. O que temos que ter sempre em mente é que a constante divulgação das informações representa uma abertura para a melhoria da inteligência coletiva mobilizada. Caros amigos, o acompanhamento das preferências de consumo causa impacto indireto na reavaliação dos índices pretendidos.

% TODO: \usepackage{graphicx} required
\begin{figure} % sem o '[H]', a figura será posicionada onde for melhor
	\centering
	\includegraphics[width=0.7\linewidth]{images/image.png}
	\caption[Figura 1]{Figura exemplo 1}
	\label{fig:fig1}
\end{figure}
% por padrão, o LaTeX posiciona a figura em um espaço onde gaste menos papel, então não será sempre onde se escolheu.

% abaixo, usando o pacote 'float' para forçar a posição:
\begin{figure}[H] % [t] 'tenta' inserir no topo, [b] na base da pg; '!' após esses, força a posição
	\centering
	\includegraphics[width=0.7\linewidth]{images/unimed2025.jpeg}
	\caption[Figura 2]{Figura exemplo 2}
	\label{fig:fig2}
\end{figure}
Considerando as lições aprendidas, a dinamização das capacidades institucionais acarreta um processo de reformulação e modernização dos compromissos firmados em instâncias multilaterais. É claro que a consolidação das estruturas nos obriga à anális`e do sistema de participação geral. À medida que avançamos, a constante divulgação das informações ainda não demonstrou convincentemente que vai participar na mudança dos níveis de motivação departamental. Levando em consideração as tendências globais, a articulação entre os diferentes níveis institucionais ressalta a relevância da participação ativa do levantamento das variáveis envolvidas. De acordo com especialistas, a contínua expansão de nossa atividade converge para práticas mais resolutivas dos conhecimentos estratégicos para atingir a excelência.

\section{Seção xyz}
Podemos já vislumbrar o modo pelo qual o acompanhamento das preferências de consumo obstaculiza a apreciação da importância dos contextos operacionais cada vez mais complexos. Evidentemente, a normatização de fluxos decisórios apresenta tendências no sentido de aprovar a manutenção das dinâmicas sociais em transformação. Com base em dados empíricos, a incorporação de perspectivas multidisciplinares impulsiona o reposicionamento institucional das rupturas provocadas pela transformação digital. Com respaldo nas evidências disponíveis, a consulta aos diversos militantes ancora-se em pressupostos teóricos consistentes dos níveis de motivação departamental. Tendo em vista as transformações em curso, a gestão eficiente dos recursos estratégicos facilita a criação dos compromissos firmados em instâncias multilaterais. \cite{CaveFaith}

À medida que avançamos, a utilização racional de ativos intangíveis ressalta a relevância da participação ativa dos modos de operação convencionais. Por conseguinte, a análise aprofundada dos indicadores-chave prepara-nos para enfrentar situações atípicas decorrentes das interfaces entre as dimensões técnico-políticas. Considerando os desafios contemporâneos, o início da atividade geral de formação de atitudes legitima a busca por soluções sistêmicas da maturidade alcançada nos processos de mudança. Por intermédio de análises qualificadas, o comprometimento entre as equipes oferece uma interessante oportunidade para verificação da governabilidade em ambientes voláteis. Nunca é demais lembrar o peso e o significado destes problemas, uma vez que a execução dos pontos do programa nos obriga à análise das formas de ação.

\section{Seção XPTO}
O cuidado em identificar pontos críticos no surgimento do comércio virtual converge para práticas mais resolutivas das posturas dos órgãos dirigentes com relação às suas atribuições. No entanto, não podemos esquecer que a expansão dos mercados mundiais representa uma abertura para a melhoria do retorno esperado a longo prazo. Acima de tudo, é fundamental ressaltar que a valorização de fatores subjetivos demanda um esforço conjunto de alinhamento do orçamento setorial. De acordo com especialistas, o comprometimento entre as equipes oferece uma interessante oportunidade para verificação dos conhecimentos estratégicos para atingir a excelência. Assim mesmo, a contínua expansão de nossa atividade é uma das consequências da fluidez dos cenários contemporâneos. \cite{KleistPiedade}

A partir de reflexões estratégicas, o novo modelo estrutural aqui preconizado obstaculiza a apreciação da importância das condições financeiras e administrativas exigidas. O empenho em analisar o desenvolvimento contínuo de distintas formas de atuação incorpora valores fundamentais à transformação das regras de conduta normativas. As experiências acumuladas demonstram que a complexidade dos estudos efetuados nos obriga à análise das novas proposições.

\subsection{Tabelas}
% para gerar tabelas, pode-se usar https://tablesgenerator.com
% Please add the following required packages to your document preamble:
% \usepackage[table,xcdraw]{xcolor}
% Beamer presentation requires \usepackage{colortbl} instead of \usepackage[table,xcdraw]{xcolor}
\begin{table}[H]
	\centering
	\sffamily
	\begin{tabular}{|l|ccc|}
		\hline
		\textbf{ID} & \multicolumn{1}{l|}{\textbf{Nome}} & \multicolumn{1}{l|}{\textbf{Sobrenome}} & \multicolumn{1}{l|}{\textbf{Idade}} \\ \hline
		\textbf{1}  & \multicolumn{1}{c|}{Ernani}        & \multicolumn{1}{c|}{Kern}               & 55                                  \\ \hline
		\textbf{2}  & \multicolumn{1}{c|}{Rodrigo}       & \multicolumn{1}{c|}{Ferro}              & 16                                  \\ \hline
		\textbf{3}  & \multicolumn{1}{c|}{Renata}        & \multicolumn{1}{c|}{Richter}            & 53                                  \\ \hline
		& \multicolumn{3}{l|}{\cellcolor[HTML]{C0C0C0}}                                                                      \\ \hline
	\end{tabular}
	\caption{Tabela 1}
	\label{tab:table1}
\end{table}

Gostaria de enfatizar que a integração das tecnologias emergentes auxilia a preparação e a composição dos índices pretendidos. Neste sentido, a adoção de políticas descentralizadoras estimula a padronização do fluxo de informações. Todavia, a necessidade de renovação processual maximiza as possibilidades por conta da maturidade alcançada nos processos de mudança. Considerando os desafios contemporâneos, o julgamento imparcial das eventualidades ilustra as tensões entre tradição e inovação das diversas correntes de pensamento. Não obstante, a gestão eficiente dos recursos estratégicos requer um olhar atento sobre os desdobramentos do remanejamento dos quadros funcionais. \cite{CaveBunny}

\section{Tabela com merge}

Abaixo uma tabela usando merge de células: 

% Please add the following required packages to your document preamble:

\begin{table}[H]
	\centering
	\sffamily
	\begin{tabular}{@{}|l|lc|@{}}
		\toprule
		\textbf{A1} & \multicolumn{2}{l|}{\multirow{2}{*}{\textbf{B1}}} \\ \cmidrule(r){1-1}
		\textbf{A2} & \multicolumn{2}{l|}{}                             \\ \midrule
		\textbf{A3} & \multicolumn{1}{c|}{}              &              \\ \bottomrule
	\end{tabular}
	\caption{Tabela 2}
	\label{tab:table2}
	\vspace{2\baselineskip} % deixa 2 linhas em branco
	
% AS LINHAS ACIMA DEIXAM A TABELA 'FEIA', COM LINHAS DESENCONTRADAS
	\renewcommand{\arraystretch}{1.5} % aumenta as margens u/d da tabela ('renewcommand' serve para várias coisas) - para voltar ao padrão, usar o valor 1.
	\begin{tabularx}{0.3\textwidth}{|X|X|X|} % usa o pacote com ajuste de margens - o 'X' define que aquela coluna será expandida
		\hline
		\centering\textbf{A1} & \multicolumn{2}{c|}{\multirow{2}{*}{\textbf{B1}}} \\ \cline{1-1}
		\hfill\textbf{A2} & \multicolumn{2}{c|}{}                             \\ \hline
		\cline{1-3}\textbf{A3} & &                                                 \\ \hline
	\end{tabularx}
	\caption{Tabela 3}
	\label{tab:table3}
\end{table}

\section{Listas}

Lista não ordenada: 
\begin{itemize}
	\item Item 1
	\item Item 2
\end{itemize}

Lista ordenada: 
\begin{enumerate}
	\item Item 1
	\item Item 2
	\begin{enumerate}
		\item Subitem 1
		\item Subitem 2
	\end{enumerate}
\end{enumerate}

Lista com descrição: 
\begin{description}
	\item[Primeiro item] Este é um item
\end{description}

\newpage % cria nova página

\section{Formatação}

\noindent\textbf{Texto em negrito}\\
\textit{Texto em itálico}\\
\underline{Texto sublinhado}\\
\textbf{\textit{\underline{Texto em negrito, itálico e sublinhado}}}\\

\begin{flushleft}
Texto à esquerda
\end{flushleft}

\begin{center}
Texto centralizado
\end{center}

\begin{flushright}
Texto à direita
\end{flushright}

{\tiny Texto pequeno.}

{\Huge Texto Grande.}

Texto Normal.
\newpage
\section{Modo Matemático}

\subsection{Equações no corpo do texto}
Equação do 2º grau: $ ax^2 + bx + c = 0 $ % equação dentro do corpo do texto

\subsection{Equações isoladas do texto}
\begin{equation} % equações ficam numeradas, para eliminar usar "*".
	x = \frac{-b \pm \sqrt{b^2 -4ac}}{2a} % \frac = fração{num}{den}, \sqrt= raiz, \sqrt[x] = raiz de x
\end{equation}

\subsection{Arrays}
\begin{equation}
	\begin{array}{cc}
		x_1= \dfrac{-b + \sqrt{b² -4ac}}{2a}, & % '_' = subescrito
		x_2= \dfrac{-b - \sqrt{b² - 4ac}}{2a} \\
		x_1= \dfrac{-b + \sqrt{b² -4ac}}{2a}, & % '_' = subescrito
		x_2= \dfrac{-b - \sqrt{b² - 4ac}}{2a} \\
	\end{array}	
\end{equation}

\subsection{Matrizes}
\begin{equation}
	A = \begin{bmatrix}
	1 & 0 & 0 \\
	0 & 1 & 0 \\
	0 & 0 & 1
	\end{bmatrix}
\end{equation}

\subsection{Seno e Tangente}
\begin{equation}
	\sin{2x}
\end{equation}
\begin{equation}
	\tan{2x}
\end{equation}

\section{Bibliografia}

Bibliografias são criadas em arquivos separados, \underline{obrigatoriamente com a extensão .bib}. 

O arquivo .bib, caso seja criado automaticamente por uma plataforma de edição, terá o seguinte formato: 

\begin{verbatim}
	@book{ID,
		author = {author},
		title = {title},
		date = {date},
		OPTeditor = {editor},
		OPTeditora = {editora},
		OPTeditorb = {editorb},
		OPTeditorc = {editorc},
		OPTtranslator = {translator},
		...
		OPTeprinttype = {eprinttype},
		OPTurl = {url},
		OPTurldate = {urldate},
}
\end{verbatim}

O 'ID' do livro é importante, e deve ser claro, pois será usado como referência dentro do documento principal. 

Todos os itens iniciados por OPT podem ser removidos ou ignorados, se não forem usados. \underline{Caso se pretenda usá-los}, é preciso remover o prefixo OPT dos campos. 

\textbf{DICA}: Para conseguir os dados dos livros 'automaticamente', pode-se ir no site da Amazon, por exemplo, copiar o número 'ISBN-10' do livro, e copiá-lo em algum site que realize a conversão 'ISBN to BibTeX', e então copiar o trecho criado. 

Exemplo abaixo: 

\begin{verbatim}
	@BOOK{CaveBunny,
		title    = "A morte de Bunny Munro",
		author   = "Cave, Nick and Morais, Fabiano",
		abstract = "A hist{\'o}ria do anti-her{\'o}i Bunny Munro, um mulherengo
		mau-car{\'a}ter que, ap{\'o}s o suic{\'\i}dio da mulher, Libby,
		parte com o filho para uma pequena odisseia rumo ao sul da
		Inglaterra e descobre que est{\'a} com os dias contados.",
		year     =  2010,
		language = "pt"
	}
	
	@BOOK{CaveFaith,
		title     = "Faith, hope and carnage",
		author    = "Cave, Nick and O'Hagan, Se{\'a}n",
		publisher = "Farrar, Straus and Giroux",
		month     =  sep,
		year      =  2022,
		language  = "en"
	}
	
	@BOOK{KleistPiedade,
		title    = "Nick Cave: piedade de mim",
		author   = "Kleist, Reinhard",
		year     =  2023,
		language = "pt"
	}
	
\end{verbatim}

O arquivo *.bib pode conter vários livros, artigos, etc.

% Bibliografia
\newpage
\addcontentsline{toc}{section}{Rererências} % adiciona as referências ao sumário
\bibliographystyle{apalike} % os estilos podem ser pesquisados na internet
\bibliography{bibaula} % define o arquivo de bibliografia

As referências só são mostradas quando houver citação dentro do texto.\\
Para incluir nas referências um livro não citado, usar: 

\begin{verbatim}
	\nocite{bookxyz}
\end{verbatim}


\end{document}
