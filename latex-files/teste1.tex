% PREÂMBULO
\documentclass[a4paper,10pt]{article}

\usepackage[utf8]{inputenc} % pacote para acentuação
\usepackage[brazil]{babel} % coloca os nomes em pt-br
\usepackage{indentfirst} % aplica indentação
\usepackage{setspace} % altera espaçamento entre linhas
%\usepackage[a4paper, margin=2cm]{geometry} % ajusta a margem geral
\usepackage[a4paper, left=2cm, right=1.5cm, top=3cm, bottom=2.5cm]{geometry} % adiciona o pacote 'geometry' e configura margens
\usepackage{xcolor} % altera as cores do documento
\usepackage{graphicx} % permite adicionar figuras, sem priorizar a posição
\usepackage{float} % insere figuras forçando o posicionamento
\usepackage{colortbl}
\usepackage{booktabs}
\usepackage{multirow} % mescla linhas de tabelas
\usepackage{tabularx} % ajusta margens
\usepackage{amsmath} % modo matemático
\usepackage{multicol} % usar texto em colunas
\usepackage{lipsum} % gerar texto de exemplo (lorem ipsum, que possui 150 parágrafos no total)
% o LIPSUM pode ser usado em português (ptlipsum), pode gerar parágrafos personalizados (SetLipsumPar{n}{parágrafo}), e pode gerar textos sobre TI, matemática, redes, etc (erlipsum). ???? dica do chatgpt... 
\usepackage{csquotes} % coloca aspas para citações diretas
\usepackage{draftwatermark} % habilita marca dágua
\usepackage{transparent} % transparência
\usepackage{enumitem} % disponibiliza opções para listas
\usepackage[colorlinks=true,linkcolor=blue,urlcolor=blue,citecolor=red]{hyperref} % ver comentários abaixo
%\usepackage[hidelinks]{hyperref} % links http - MANTER ABAIXO DOS OUTROS 'usepackage', pois pode alterar alguns comandos. - HIDELINKS evita que os títulos do sumário e notas de rodapé recebam um círculo colorido

\setlength{\parindent}{1,5cm} % ajusta a indentação do parágrafo
\setlength{\parskip}{0.1cm} % ajusta a separação entre parágrafos
\setlength{\columnsep}{5mm} % separação entre colunas
\setlist{itemsep=-5pt, topsep=0pt} % define espaçamento de listas globalmente

\date{27/11/2025}

\renewcommand{\sin}{\mathrm{sen\hspace{0.5mm}}} % traduz a função 'sin' para o português
%\DeclareMathOperator{\sen}{sen} % método melhor que o acima, mas exige que se use '\sen' no texto

% CORPO DO TEXTO
\begin{document}
\SetWatermarkText{} % seta o texto para a marca dágua antes da posição da figura, ou elimina, se deixado em branco; se não declarado, usar a marca 'draft'
\sffamily
\title{\textbf{\huge{Capa e Sumário}}}
\author{\textbf{Ernani Kern}}
\date{\textbf{Dezembro de 2025}}
\maketitle % mostra os parâmetros acima, senão, não aparecem
\thispagestyle{empty} % oculta a numeração da página
\newpage

\setcounter{page}{1} % reseta a contagem de páginas, e inicia aqui com '1'
\pagenumbering{Roman} % altera para alg romano; o 'R' maiúsculo deixa a numeração em maiúsculas
\tableofcontents % cria o sumário
\newpage

\listoffigures % cria sumário de figuras
\newpage

\listoftables % cria a lista de tabelas
\newpage

\setcounter{page}{1} % reinicia de novo a numeração de páginas
\pagenumbering{arabic} % numeração de páginas em arábico
\section{Seção antes da seção 1}
\onehalfspacing % espaçamento 1,5 entre linhas
\lipsum[1-2] % parágrafos 1 e 2 do lorem ipsum

\textcolor{red}{Texto em vermelho}
\colorbox{gray}{\textcolor{blue}{Texto Azul e Fundo cinza}}
%\pagecolor{cyan}

\section{Seção 1}
\setstretch{2} % ajusta espaçamento personalizado
\lipsum[3-4] % parágrafos 3 e 4 do lorem ipsum

% para abrir parágrafo, deixar linha anterior em branco

\begin{onehalfspace} % \begin{spacing}{1,5}
\lipsum[5]
\end{onehalfspace} % \end{spacing}

\subsection{Subseção x.x}
\SetWatermarkText{\transparent{0.3}\includegraphics[width=0.5\linewidth]{images/marca.jpg}} % inclui marca dágua a partir deste ponto
\lipsum[6]
% o '\\' gera quebra de linha

\subsubsection*{Subseção x.x.x}
\lipsum[7]
% o '*' na '\subsubsection' elimina a numeração da seção; se houver nova seção após, a numeração continuará a partir da última numerada.

\subsubsection{Subseção y.y.y}
\lipsum[8-9]

% TODO: \usepackage{graphicx} required
\begin{figure} % sem o '[H]', a figura será posicionada onde for melhor
	\centering
	\includegraphics[width=0.7\linewidth]{images/image.png}
	\caption[Figura 1]{Figura exemplo 1}
	\label{fig:fig1}
\end{figure}
% por padrão, o LaTeX posiciona a figura em um espaço onde gaste menos papel, então não será sempre onde se escolheu.

% abaixo, usando o pacote 'float' para forçar a posição:
\begin{figure}[H] % [t] 'tenta' inserir no topo, [b] na base da pg; '!' após esses, força a posição
	\centering
	\includegraphics[width=0.7\linewidth]{images/unimed2025.jpeg}
	\caption[Figura 2]{Figura exemplo 2}
	\label{fig:fig2}
\end{figure}
\lipsum[10]\footnote{Esta é a nota de rodapé 1.}\footnote{Esta é a nota de rodapé 2.}

\SetWatermarkText{RASCUNHO} % muda a marca dágua a partir daqui
\SetWatermarkScale{3}

\begin{multicols}{2} % texto em duas colunas
\section{Seção xyz}
\lipsum[11-14] \cite{CaveFaith}

\section{Seção XPTO}
\lipsum[15-17]
Citação ao Reinhart Kleist. \cite{KleistPiedade}

\subsection{Tabelas}
% para gerar tabelas, pode-se usar https://tablesgenerator.com
% Please add the following required packages to your document preamble:
% \usepackage[table,xcdraw]{xcolor}
% Beamer presentation requires \usepackage{colortbl} instead of \usepackage[table,xcdraw]{xcolor}
\begin{table}[H]
	\centering
	\sffamily
	\begin{tabular}{|l|ccc|}
		\hline
		\textbf{ID} & \multicolumn{1}{l|}{\textbf{Nome}} & \multicolumn{1}{l|}{\textbf{Sobrenome}} & \multicolumn{1}{l|}{\textbf{Idade}} \\ \hline
		\textbf{1}  & \multicolumn{1}{c|}{Ernani}        & \multicolumn{1}{c|}{Kern}               & 55                                  \\ \hline
		\textbf{2}  & \multicolumn{1}{c|}{Rodrigo}       & \multicolumn{1}{c|}{Ferro}              & 16                                  \\ \hline
		\textbf{3}  & \multicolumn{1}{c|}{Renata}        & \multicolumn{1}{c|}{Richter}            & 53                                  \\ \hline
		& \multicolumn{3}{l|}{\cellcolor[HTML]{C0C0C0}}                                                                      \\ \hline
	\end{tabular}
	\caption{Tabela 1}
	\label{tab:table1}
\end{table}

\end{multicols}

\section{Citações Internas}

A figura \ref{fig:fig2} está sendo referenciada aqui.

A tabela \ref{tab:table3} está sendo referenciada aqui.

\section{Tabela com merge}

Abaixo uma tabela usando merge de células: 

% Please add the following required packages to your document preamble:

\begin{table}[H]
	\centering
	\sffamily
	\begin{tabular}{@{}|l|lc|@{}}
		\toprule
		\textbf{A1} & \multicolumn{2}{l|}{\multirow{2}{*}{\textbf{B1}}} \\ \cmidrule(r){1-1}
		\textbf{A2} & \multicolumn{2}{l|}{}                             \\ \midrule
		\textbf{A3} & \multicolumn{1}{c|}{}              &              \\ \bottomrule
	\end{tabular}
	\caption{Tabela 2}
	\label{tab:table2}
	\vspace{2\baselineskip} % deixa 2 linhas em branco
	
% AS LINHAS ACIMA DEIXAM A TABELA 'FEIA', COM LINHAS DESENCONTRADAS
	\renewcommand{\arraystretch}{1.5} % aumenta as margens u/d da tabela ('renewcommand' serve para várias coisas) - para voltar ao padrão, usar o valor 1.
	\begin{tabularx}{0.3\textwidth}{|X|X|X|} % usa o pacote com ajuste de margens - o 'X' define que aquela coluna será expandida
		\hline
		\centering\textbf{A1} & \multicolumn{2}{c|}{\multirow{2}{*}{\textbf{B1}}} \\ \cline{1-1}
		\hfill\textbf{A2} & \multicolumn{2}{c|}{}                             \\ \hline
		\cline{1-3}\textbf{A3} & &                                                 \\ \hline
	\end{tabularx}
	\caption{Tabela 3}
	\label{tab:table3}
\end{table}

\section{Listas}

Lista não ordenada: 
\begin{itemize}
	\item Item 1
	\item Item 2
\end{itemize}

Lista ordenada: 
\begin{enumerate}
	\item Item 1
	\item Item 2
	\begin{enumerate}
		\item Subitem 1
		\item Subitem 2
	\end{enumerate}
\end{enumerate}

Lista com descrição: 
\begin{description}
	\item[Primeiro item] Este é um item
	\item[Segundo item] Item dois
\end{description}

\newpage % cria nova página

\section{Formatação}

\noindent\textbf{Texto em negrito}\\
\textit{Texto em itálico}\\
\underline{Texto sublinhado}\\
\textbf{\textit{\underline{Texto em negrito, itálico e sublinhado}}}\\

\begin{flushleft}
Texto à esquerda
\end{flushleft}

\begin{center}
Texto centralizado
\end{center}

\begin{flushright}
Texto à direita
\end{flushright}

{\tiny Texto pequeno.}

{\Huge Texto Grande.}

Texto Normal.
\newpage
\SetWatermarkText{}
\section{Modo Matemático}

\subsection{Equações no corpo do texto}
Equação do 2º grau: $ ax^2 + bx + c = 0 $ % equação dentro do corpo do texto

\subsection{Equações isoladas do texto}
\begin{equation} % equações ficam numeradas, para eliminar usar "*".
	x = \frac{-b \pm \sqrt{b^2 -4ac}}{2a} % \frac = fração{num}{den}, \sqrt= raiz, \sqrt[x] = raiz de x
\end{equation}

\subsection{Arrays}
\begin{equation}
	\begin{array}{cc}
		x_1= \dfrac{-b + \sqrt{b² -4ac}}{2a}, & % '_' = subescrito
		x_2= \dfrac{-b - \sqrt{b² - 4ac}}{2a} \\
		x_1= \dfrac{-b + \sqrt{b² -4ac}}{2a}, & % '_' = subescrito
		x_2= \dfrac{-b - \sqrt{b² - 4ac}}{2a} \\
	\end{array}	
\end{equation}

\subsection{Matrizes}
\begin{equation}
	A = \begin{bmatrix}
	1 & 0 & 0 \\
	0 & 1 & 0 \\
	0 & 0 & 1
	\end{bmatrix}
\end{equation}

\subsection{Seno e Tangente}
\begin{equation}
	\sin{2x}
\end{equation}
\begin{equation}
	\tan{2x}
\end{equation}

\section{Bibliografia}

Bibliografias são criadas em arquivos separados, \underline{obrigatoriamente com a extensão .bib}. 

O arquivo .bib, caso seja criado automaticamente por uma plataforma de edição, terá o seguinte formato: 

\begin{verbatim}
	@book{ID,
		author = {author},
		title = {title},
		date = {date},
		OPTeditor = {editor},
		OPTeditora = {editora},
		OPTeditorb = {editorb},
		OPTeditorc = {editorc},
		OPTtranslator = {translator},
		...
		OPTeprinttype = {eprinttype},
		OPTurl = {url},
		OPTurldate = {urldate},
}
\end{verbatim}

O 'ID' do livro é importante, e deve ser claro, pois será usado como referência dentro do documento principal. 

Todos os itens iniciados por OPT podem ser removidos ou ignorados, se não forem usados. \underline{Caso se pretenda usá-los}, é preciso remover o prefixo OPT dos campos. 

\textbf{DICA}: Para conseguir os dados dos livros 'automaticamente', pode-se ir no site da Amazon, por exemplo, copiar o número 'ISBN-10' do livro, e copiá-lo em algum site que realize a conversão 'ISBN to BibTeX', e então copiar o trecho criado. 

Exemplo abaixo: 

\begin{verbatim}
	@BOOK{CaveBunny,
		title    = "A morte de Bunny Munro",
		author   = "Cave, Nick and Morais, Fabiano",
		abstract = "A hist{\'o}ria do anti-her{\'o}i Bunny Munro, um mulherengo
		mau-car{\'a}ter que, ap{\'o}s o suic{\'\i}dio da mulher, Libby,
		parte com o filho para uma pequena odisseia rumo ao sul da
		Inglaterra e descobre que est{\'a} com os dias contados.",
		year     =  2010,
		language = "pt"
	}
	
	@BOOK{CaveFaith,
		title     = "Faith, hope and carnage",
		author    = "Cave, Nick and O'Hagan, Se{\'a}n",
		publisher = "Farrar, Straus and Giroux",
		month     =  sep,
		year      =  2022,
		language  = "en"
	}
	
	@BOOK{KleistPiedade,
		title    = "Nick Cave: piedade de mim",
		author   = "Kleist, Reinhard",
		year     =  2023,
		language = "pt"
	}
	
\end{verbatim}

O arquivo *.bib pode conter vários livros, artigos, etc.

\section{Macros}

Trigger: usado para substituição de termos

As macros podem ser de três tipos:

\begin{enumerate}
	\item Normal: 
	\subitem Substituir textos curtos por longos
	\subitem Substituir abreviaturas por comandos longos
	\item Environment: 
	\subitem Cria o 'environment' desejado, com o 'begin\{\} end\{\}'
	\subitem Pode ser necessário usar \begin{verbatim}'\n'\end{verbatim}para quebra de linha
	\item Scripts
	\subitem Para gerar scripts para diversas funções
\end{enumerate} 

\section{Segmentando o documento em vários arquivos}

A divisão do documento em arquivos separados é útil para organização e caso se queira compartilhar apenas alguns trechos, como algum capítulo, introdução, etc. Os arquivos segmentados também devem ter a extensão .tex.

Neste caso, um documento será o principal, e conterá o preâmbulo, os comandos 'begin\{document\}, etc. 

Os arquivos segmentados não terão preâmbulo ou 'begin\{document\}', e por isso não podem ser compilados diretamente, senão darão erro. 

Estes arquivos devem ser compilados através do arquivo principal, através de:\begin{verbatim}\input{file}\end{verbatim}

Os arquivos segmentados podem, entretanto, declarar outros comandos, para definição de tabelas, inserção de figuras, negrito, itálico, etc. 

\lipsum[1]
\textbf{Este parágrafo está em um arquivo separado.}

\section{Citações Diretas}

"Esta é uma citação direta literal, mas o primeiro par de aspas está invertido"

''Aqui usei duas aspas simples no início, não resolveu."

\enquote{Outro texto entre aspas, usando o pacote 'csquote', funciona.}

\begin{quote}
	Uma citação curta usando 'begin\{quote\}'.
\end{quote}

\begin{quotation}
	Citação mais longa, agora usando 'begin\{quotation\}', este modo adiciona indentações para os textos mais longos e profundos.... 
\end{quotation}

\section{Documentos Formatados pela ABNT}

Site da \href{https://www.abntex.net.br/}{AbnTeX}: \url{https://www.abntex.net.br/}

O \href{https://www.abntex.net.br}{abnTeX2} é um conjunto de classes e pacotes LaTeX criado para facilitar a produção de documentos formatados segundo as normas da ABNT (trabalhos acadêmicos, relatórios, livros, monografias, TCCs, teses e dissertações).

Ele não é só um template, mas um framework completo para documentos em português seguindo ABNT.

O abnTeX inclui: 

\begin{itemize}
	\item abntex2 (genérica)
	\item abntex2-report (para relatórios)
	\item abntex2-article (para artigos)
	\item abntex2-book
	\item abntex2-tese (para teses)
\end{itemize}

O site do abnTeX disponibiliza um instalador que fornece arquivos .cls, .sty e templates para o texlive. 

% Bibliografia
\newpage
\addcontentsline{toc}{section}{Referências} % adiciona as referências ao sumário / 'toc' = table of contents
\bibliographystyle{apalike} % os estilos podem ser pesquisados na internet
\bibliography{bibaula} % define o arquivo de bibliografia

As referências só são mostradas quando houver citação dentro do texto.\\
Para incluir nas referências um livro não citado, usar: 

\begin{verbatim}
	\nocite{bookxyz}
\end{verbatim}

\end{document}
